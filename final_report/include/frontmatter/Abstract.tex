% CREATED BY DAVID FRISK, 2016
Accessibility of Teaching Materials\\
Exploring Obtainability and Testing Usability in Design of Shareable Teaching Materials\\
HÅKAN ANDERSSON \& SEBASTIAN EVERETT ERIKSSON\\
Department of Communication and Learning in Science\\
Chalmers University of Technology \setlength{\parskip}{0.5cm}

\thispagestyle{plain}			% Suppress header 
\setlength{\parskip}{0pt plus 1.0pt}
\section*{Abstract}
For shareable teaching materials to work as intended, they need to be accessible to possible recipients. In this study, accessibility is defined as being obtainable and usable. \\[0.5cm]
The obtainability aspect is primarily explored via literature study. The usability aspect is analyzed by testing of existing teaching materials. The methodology is inspired by usability testing methods found in computer science and IT. \\[0.5cm]
Research questions created to be answered in this thesis are:
\begin{description}
    \item $\bullet$ RQ1: How can usability testing be used to improve the usability of teaching mate-rials?
    \item $\bullet$ RQ2: How can usability testing as a method be made accessible for teachers with limited experience of usability design?
    \item $\bullet$ RQ3: What factors do teachers consider when deciding on how to use a teaching material?
    \item $\bullet$ RQ4: From the perspective of a technological system, how can usability design for teaching materials be used to help teachers?
\end{description}


% KEYWORDS (MAXIMUM 10 WORDS)
\vfill
Keywords: usability, obtainability, teaching materials, accessibility, Kleindagarna, Steve Krug.

\newpage				% Create empty back of side
\thispagestyle{empty}
\mbox{}