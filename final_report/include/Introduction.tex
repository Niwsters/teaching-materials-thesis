% CREATED BY DAVID FRISK, 2016
\chapter{Introduction}

Teachers often design their own teaching materials. Many problems may appear when sharing these materials with each other, including: misunderstood abbreviations, unclear purpose and structure, and lack of adaptability that require time and resources for reworking the material.

Usability testing is a method used in software development to discover usability problems in a cheap, efficient and easy-to-do manner. As the usability testing proponent Krug proposes [KÄLLA], these tests can be used for other projects outside of software development. This study explores the use of usability tests in the context of teaching materials, and how it can be done effectively, even by teachers without previous knowledge of usability design.

\section{Research Questions}
More specifically, this study aims to answer the research questions below:

\begin{enumerate}
	\item How can usability testing be adopted to improve the usability of teaching materials?
	\item How can usability testing as a method be made accessible for teachers with limited experience of usability design?
	\item What factors do teachers consider when deciding on how to use a teaching material?
	\item From the perspective of a technological system, how can usability design for teaching materials be used to help teachers?
\end{enumerate}

\section{Background}
[Kleindagarna, what more?]

\subsection{Kleindagarna}
To be able to test teaching materials, a teaching material must exist. For this study, all the teaching materials tested were created by a triannual three-day workshop called Kleindagarna. Kleindagarna is organized by the Swedish Committee for Mathematics Education (SKM), the Swedish National Committee for Mathematics (KVA), the Institute Mittag-Leffler and is funded by Brummer \& Partners.

At this workshop, maths teachers from upper secondary school meet up with professors and maths teachers from universities and colleges at the Mittag-Leffler Institute outside Stockholm. During the workshop they collaborate to produce teaching materials in mathematics. These teaching materials are meant to be used for teaching upper secondary school, and often touch subjects that are not typically found in course literature.[SOURCE: Kleindagarna.se]
