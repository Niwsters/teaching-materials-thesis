\chapter{Discussion}
\section{Required adaptability depends on the teacher's autonomy}
\todo{Some speculations that I decided to write down to see where the final report might end up. Better make risky speculations early rather than late, so we get a chance to revise them. /S}
The effects of teaching materials on teacher workload is complex. A lot boils down to the autonomy in which the teacher can choose and work with a material.

If a teacher is forced to use a material, either due to rules or due to external factors, the material might have negative consequences. But if a teacher has the right to use or not use a material in the way they want, they can choose to use the parts that help them, and ignore the parts that hold them back. For example, "I use the book for simplicity's sake, but when something requires extra attention I find a material that better suits the students' needs."

How materials are used in a school is part of the school's material technology. Teachers and students both learn habits that affect the rest of the school. Thus, the school as a whole should consider how to best make use of materials as a resource: Do they have access to new material when needed and/or wanted, and do they have a stable method to fall back on when the materials produce too much workload? Note that being forced to use some new and different material is just as negative on the teacher's autonomy as being forced to do like everyone else.

\section{Obtainability}
\todo{this could possibly be the first paragraph of this section in the final report. It is phrased as it aims to answer an RQ, so maybe it shouldn't be in discussion(?) /H}
Words like obtainable and teaching materials are broad by definition, and school as an institution is complex by nature. This discussion on obtainability can therefore be expected to fail at giving a complete explanation to how these words fit together, but it will try to answer some of the difficulties teachers are facing in obtaining teaching materials.
