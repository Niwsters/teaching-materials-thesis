\chapter{Discussion}

\section{Answers to RQ1: How can usability testing be used to improve the usability of teaching materials?}

Looking at what issues were found during the usability tests in this study, usability testing of teaching materials seemed to work similarly to how they work with websites, according to Krug's method. Both methods are effective in finding problems with misunderstandings and lack of clarity. However, when it comes to testing how the teachers would choose and use their material, the results are dependent on what is possible in a realistic teaching situation. For example, as described in [REFERENS: ACCOUNTING RO TEACHERS' AND STUDENTS' PREVIOUS KNOWLEDGE], some teachers evaluated a material according to their students' previous knowledge. Similar results were also found relating to how well a material connected to the school curriculum [REFERENS?]. Without previous experience with working as a teacher, such things could be difficult to evaluate. That said, this study did not directly test and compare doing usability testing done with non-teachers, so specifying what difference having non-teachers as test subjects would make for the results is difficult to say. Studying what test subjects would be eligible for testing different aspects of teaching materials is an interesting subject for further study.

Aside from finding common usability problems, the usability tests also produced a few results that were unique for teaching materials, compared to testing websites or other things. For example, student handouts specifically seemed appreciated in many tests [REFERENS: HAVING STUDENT HANDOUTS AS PART OF A MATERIAL IS APPRECIATED]. This likely has several reasons:

\begin{itemize}
	\item Student handouts require little preparation to use, since they often simply require printing, in contrast to for example having to write a slideshow from scratch.
	\item Student handouts are concrete and easy to understand, as long as the teacher understands that they are student handouts. This is because teachers know what student handouts are, and they know intuitively how they are supposed to be used, compared to more abstract materials.
\end{itemize}

Even if many materials could be improved with student handouts, it is likely not an all-encompassing solution. Similarly, future usability tests might find problems with student handouts that this study did not find. However, the result shows a strength in usability testing teaching materials, in that similar findings might be possible in more tests. Furthermore, the finding says something about the importance of being concrete: If a teaching material is to explain something abstract, having an example of what an explanation to a student would look like might make the explanation easier to understand for the teacher. Such realizations are a reason why the usability testing also makes the tester a better usability designer, aside from finding specific problems for specific materials.

Finally, testing the materials list as well as the materials in it showed an important distinction: The difference between making good content, and making the content easier to understand. While solving common usability problems is an important part of designing good teaching materials, it's also important to consider how useful a materials' theme or content is. For example, in Sample case 2 [REFERENS: Sample case 2. Kleinmaterial: Vanliga missuppfattningar] it was revealed that the material was appreciated, despite big usability problems. The teacher expressed interest in the material's theme, which was common misconceptions in mathematics. The material also had student handouts. Usability testing can likely be used to find whether or not a teacher appreciates a specific theme, but figuring out what themes teachers are looking for, amon gother things, can be difficult to do by testing only one material. In other words, usability testing does not seem effective in testing whether a material asks the right questions, but it is effective in testing whether it answers its questions clearly.

\section{Answers to RQ2: How can usability testing as a method be made accessible for teachers with limited experience of usability design?}

- simplifying: explaining Krug's method, using our u.t.-script. How to use it early in the design process. [MINUS THE MATERIALS LIST TESTING?]

- testing on colleagues
- understanding the difference between asking, "what do you think?", and actually observing someone using the material live.

- working with small iterations: sloppy design, perfectionistic testing. Test as early as possible.

\section{Answers to RQ3: What factors do teachers consider when deciding on how to use a teaching material?}

- according to demands from the educational system: connection to curriculum, time, workload, what they are working on now.
- student needs; previous knowledge and/or motivation.
- how clearly the material is described in "the list;" What it is, how it is used. They want to know what they're getting.

\section{Answers to RQ4: From the perspective of a technological system, how can usability design for teaching materials be used to help teachers?}

- help how? workload, quality of teaching, collaboration between teachers and teachers, and teachers and other material designers.
- making U.T. part of teaching material technology, in a holistic manner: making teachers better usability designers rather than prescribing "the way to design materials." (Avoid institutionalizing usability testing)
- increasing teacher confidence vs. making teachers feel like bad designers.
- negative effects on innovation: what if Kleindagarna limited themselves to the curriculum?

\section{Required adaptability depends on the teacher's autonomy}
\todo{Some speculations that I decided to write down to see where the final report might end up. Better make risky speculations early rather than late, so we get a chance to revise them. /S}
The effects of teaching materials on teacher workload is complex. A lot boils down to the autonomy in which the teacher can choose and work with a material.

If a teacher is forced to use a material, either due to rules or due to external factors, the material might have negative consequences. But if a teacher has the right to use or not use a material in the way they want, they can choose to use the parts that help them, and ignore the parts that hold them back. For example, \textit{"I use the book for simplicity's sake, but when something requires extra attention I find a material that better suits the students' needs."}

How materials are used in a school is part of the school's material technology. Teachers and students both learn habits that affect the rest of the school. Thus, the school as a whole should consider how to best make use of materials as a resource: Do they have access to new material when needed and/or wanted, and do they have a stable method to fall back on when the materials produce too much workload? Note that being forced to use some new and different material is just as negative on the teacher's autonomy as being forced to do like everyone else.

\section{Obtainability}
\todo{this could possibly be the first paragraph of this section in the final report. It is phrased as it aims to answer an RQ, so maybe it shouldn't be in discussion(?) /H}
Words like obtainable and teaching materials are broad by definition, and school as an institution is complex by nature. This discussion on obtainability can therefore be expected to fail at giving a complete explanation to how these words fit together, but it will try to answer some of the difficulties teachers are facing in obtaining teaching materials.

\section{Limitations of the study}
\subsection{Homogeneity of usability test data}
The study's focus could be different if the test data collected was more homogeneous. A good way to achieve this would have been to collaborate with teachers from the same school. Regular testing with the same teachers could then be established, which better reflect the way this study's findings are proposed to be used. Interest was showed by representatives on schools contacted, but claims were made that the teachers' schedules didn't allow for this kind of collaboration. 
Any effects exclusive to collaboration within a school's teacher base has therefore not been examined. The usability tests was instead exclusively tested on math teachers and students nearly eligible to teach maths. Even though the data less homogeneous, it can be assumed this has led to findings that would not have been made when testing exclusively teachers of a single school.

\subsection{Control of usability test data}
In this study, exploring accessibility has encompassed studying both obtainability and usability. To collect more data on obtainability, usability tests were preceded by the usability subjects choosing the teaching material they wanted to test. This reduced the authors' control of testing certain teaching materials. 
This removed the ability to focus on a particular teaching material and usually prevented the authors to get the desired familiarity with the teaching material ahead of time.

\subsection{Balancing revision sizes}
\todo{Should this be included in conclusions instead? /H}
Usability testing makes it possible to revise the tested teaching material. Much of the data collected during the testing figuratively screams to be put to use in a revision. Adding alteration to a teaching material is in itself a project. When tackling a project, one should have a plan on how to reach the goal. Seeing that the development cycle in this study is iterative, adopting an iterative development process when making a revision is appropriate. 
That means that one should make many smaller changes, instead of trying to implement everything at once. A benefit of this is that one can have a new, albeit smaller, revision ready each time the teaching material is put down. This smaller revision is then ready to be usability tested on, if such an opportunity emerges. 

\section{Future work}	
\subsection{For teachers}
A suggestion is that teachers can use the methods presented in this thesis as activities in collaborative meetings, as a way to assess the accessibility of teacher materials. They could also discuss how they are affected by accessibility of teacher materials created by others. To assess accessibility on an institutional level, the following questions could be asked:
\begin{description}
\item $\bullet$ In what ways might their economy deny better quality education (obtainability issue)? 
\item $\bullet$ Is the quality of their education unreasonably dependent on the teachers finding teaching materials themselves (obtainability issue)? 
\item $\bullet$ Do their teachers use teaching materials that can not be shared to others, e.g. substitute teachers, without a significant drop in educational quality (usability issue)?
\item $\bullet$ Do their teachers produce their own material with the sole intent of only using it themselves (obtainability and usability issue)?
\end{description}
\subsection{For universities and colleges}
Future theses could be made to e.g. verify, falsify, implement, expand and/or improve upon this thesis.
\subsection{For others}
Other fields of study could adopt a usability testing method, perhaps one inspired by the iterative method developed in this thesis, to identify the unknown in their particular field. 
