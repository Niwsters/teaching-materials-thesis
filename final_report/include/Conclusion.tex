\section{Answers to research questions}

\subsection{RQ1: How can usability testing be used to improve the usability of teaching materials?}
By usability testing a teaching material, feedback from a teacher is collected. This feedback can then be 
used as input when revising the teaching material. This process can be repeated, both to validate that the 
revision improved the material and to collect further input for future revisions.

\subsection{RQ2: How can usability testing as a method be made accessible for teachers with limited experience of 
usability design?}
A usability testing script in a suitable language can be presented to teachers. Such a script involves 
everything a teacher is supposed to do and say, making it easy to follow and execute without any preparations.

\subsection{RQ3: What factors do teachers consider when deciding on how to use a teaching material?}
[data from usability tests, obtainability]

\subsection{RQ4: From the perspective of a technological system, how can usability design for teaching materials 
be used to help teachers?}
[har S någon särskild tanke med denna?] 